%===============================================================================
% (c) Dominik Harmim


%===============================================================================



\chapter{Contents of the Attached Memory Media}
\label{app:memMedia}

This appendix lists the contents of the attached memory media. In particular, the attached memory media contains the following:
\begin{itemize}
    \item \texttt{\textbf{/xharmi00-thesis-2021.pdf}}
        \begin{itemize}
            \item This thesis in PDF format.
        \end{itemize}

    \item \texttt{\textbf{/thesis-latex/}}
        \begin{itemize}
            \item \LaTeX\ source files of the thesis.
            
            \item It can be compiled into PDF format using \texttt{pdflatex} and \texttt{bibtex} with the \texttt{make} command (\texttt{Makefile} is included). However, the attached PDF file has been compiled in Overleaf.
        \end{itemize}

    \item \texttt{\textbf{/README.md}}, \texttt{\textbf{/README.html}}
        \begin{itemize}
            \item Files containing (among other things) an installation and user manual. It is available in both Markdown and HTML formats.
        \end{itemize}
    
    \item \texttt{\textbf{/src/}}
        \begin{itemize}
            \item The source code of the Facebook Infer framework with the Atomer plugin.
            
            \item Atomer is \uv{hooked} in several files of the framework. Nevertheless, the majority of the implementation is located in the sub-directory \texttt{infer/src/atomicity/}.
        \end{itemize}
    
    \item \texttt{\textbf{/infer-linux-x86-64-atomer-v2.0.0/}}
        \begin{itemize}
            \item Compiled Facebook Infer with the new version of Atomer (v2.0.0). It has been compiled on the Debian GNU/Linux x86-64 10.9 (Buster) operating system.
            
            \item Executable binaries are available in the sub-directory \texttt{bin/}.
        \end{itemize}
    
    \item \texttt{\textbf{/examples/}}
        \begin{itemize}
            \item Some example programs with reference outputs of the analysis. These programs have been used for experimental testing of the correctness of the analysis.
        \end{itemize}
\end{itemize}



%===============================================================================



\chapter{Installation and User Manual}
\label{app:man}

This appendix serves as an installation and user manual. An extended manual (together with some examples) can be found at \emph{Atomer's Wiki} (\url{https://github.com/harmim/infer/wiki}) and on the attached memory media (see Appendix~\ref{app:memMedia}).

Further, it is assumed that the \emph{working directory} contains all the files from the attached memory media, as described in Appendix~\ref{app:memMedia}. If not, the \texttt{/src/} directory may be cloned using \texttt{git} through \texttt{SSH} or \texttt{HTTPS} using one of the following commands:
\begin{lstlisting}[style=bash]
<@\textcolor{magenta}{git}@> clone git@github.com:harmim/infer.git --branch atomer-v2.0.0 --single-branch --depth 1 --recurse-submodules src
<@\textcolor{magenta}{git}@> clone https://github.com/harmim/infer.git --branch atomer-v2.0.0 --single-branch --depth 1 --recurse-submodules src
\end{lstlisting}
Alternatively, it can be downloaded using \texttt{curl} as follows:
\begin{lstlisting}[style=bash]
curl -sL https://github.com/harmim/infer/archive/refs/tags/atomer-v2.0.0.tar.gz <@|~\textcolor{magenta}{tar}@> -xz
mv infer-atomer-v2.0.0 src
\end{lstlisting}
If the \texttt{/infer-linux-x86-64-atomer-v2.0.0/} directory containing executable binaries is not present, it can be downloaded using \texttt{curl} as follows:
\begin{lstlisting}[style=bash]
mkdir infer-linux-x86-64-atomer-v2.0.0
curl -sL <@\scriptsize{https://github.com/harmim/infer/releases/download/atomer-v2.0.0/infer-atomer-v2.0.0.tar.xz}@> <@|~\textcolor{magenta}{tar}@> -xJ -C infer-linux-x86-64-atomer-v2.0.0
\end{lstlisting}


\section*{Installation Manual}

In the \texttt{/infer-linux-x86-64-atomer-v2.0.0/} directory, there is Facebook Infer with Atomer already compiled. Thus, it can be directly used. In the sub-directly \texttt{bin/}, there are executable binaries. Manual pages can be found in the sub-directory \texttt{share/man/}. The binaries have been compiled for the Debian GNU/Linux x86-64 10.9 (Buster) operating system. It can be installed system-wide using the following commands:
\begin{lstlisting}[style=bash]
PATH="infer-linux-x86-64-atomer-v2.0.0/bin:$PATH"
MANPATH="infer-linux-x86-64-atomer-v2.0.0/share/man:$MANPATH"
\end{lstlisting}

Another option is to build Atomer from the source or to run it in Docker. These options are discussed in the following sections. \textbf{Running pre-compiled binaries in Docker is a~recommended and quick method.}

\subsection*{Building Atomer from the Source}

\textit{Building Atomer from the source may be very time-consuming.}

At first, it is required to install Facebook Infer's dependencies and then to compile Facebook Infer with Atomer. Here are the prerequisites to be able to compile Facebook Infer with Atomer on Linux:
\begin{itemize}
    \item \texttt{opam} $ \geq $ 2.0.0,
    
    \item \texttt{pkg-config},
    
    \item Java (only needed for the Java analysis),
    
    \item \texttt{gcc} $ \geq $ 5.X or \texttt{clang} $ \geq $ 3.4 (only needed for the C/C++ analysis),
    
    \item \texttt{autoconf} $ \geq $ 2.63,
    
    \item \texttt{automake} $ \geq $ 1.11.1,
    
    \item \texttt{cmake} (only needed for the C/C++ analysis).
\end{itemize}
See \texttt{/src/docker/atomer-v2.0.0/Dockerfile} for inspiration on how to install the dependencies on Linux. It includes the dependencies needed to build Facebook Infer with Atomer on Debian 10 (Buster).

The installation of Facebook Infer with Atomer can be done using the following commands:
\begin{lstlisting}[style=bash]
cd src
./build-infer.sh
sudo make install
\end{lstlisting}
The official Facebook Infer's installation manual (which also includes instructions for other operating systems) can be found at \url{https://github.com/harmim/infer/blob/atomer-v2.0.0/INSTALL.md}.

Facebook Infer with Atomer now should be installed system-wide. Executable binaries are located in \texttt{/src/infer/bin/}.

\subsection*{Running Atomer in Docker}

Alternatively, Facebook Infer with Atomer can be easily built and run in Docker. The installation is much simpler and faster. At first, it is needed to install \emph{Docker Engine}. It is available on a~variety of Linux platforms, macOS, and Windows~10. Get it and follow the instructions from \url{https://docs.docker.com/engine/install}.

The first option is to build Facebook Infer with Atomer from the source in Docker. There is prepared a~Docker image for it. \textit{Building the image can take a~few hours}. The image is based on Debian Linux. After running a~container built from this image, Facebook Infer with Atomer will be compiled and installed system-wide in the container. The image can be built and the container run using the command below:
\begin{lstlisting}[style=bash]
./src/docker/atomer-v2.0.0/run.sh # there is also run.bat for Windows
\end{lstlisting}

The second option is to run \emph{pre-compiled binaries} of Facebook Infer with Atomer in Docker. There is prepared another Docker image for it. \textit{Building the image can take, at most, a~few minutes.} The image is based on Debian Linux as well. After running a~container built from this image, binaries of Facebook Infer with Atomer will be downloaded and installed system-wide in the container. The image can be built and the container run using the command below:
\begin{lstlisting}[style=bash]
./src/docker/atomer-v2.0.0-bin/run.sh # there is also run.bat for Windows
\end{lstlisting}


\section*{User Manual}

\textit{This section assumes that Facebook Infer with Atomer is installed system-wide and executable by the command} \texttt{infer}.

In general, to analyse a~C/C++ program with Facebook Infer, it can be done using the following command (for a~single source file):
\begin{lstlisting}[style=bash]
<@\textcolor{magenta}{infer}@> -- gcc -c source_file.c
\end{lstlisting}
Java programs can be analysed using the following command:
\begin{lstlisting}[style=bash]
<@\textcolor{magenta}{infer}@> -- javac source_file.java
\end{lstlisting}
Another option is to analyse the entire project with \texttt{Makefile} using the following:
\begin{lstlisting}[style=bash]
<@\textcolor{magenta}{infer}@> -- make <target>
\end{lstlisting}
For advanced usage, see \url{https://fbinfer.com/docs/infer-workflow}. Many other build systems may be used; see \url{https://fbinfer.com/docs/analyzing-apps-or-projects}.

Atomer is \emph{deactivated} by default. The analysis has to be executed twice (it has two phases). Each phase runs with a~different command-line argument. The first phase derives \emph{sets of functions that should be executed atomically} into a~file \texttt{atomic-sets}. The second phase then \emph{detects atomicity violations} according to this file. So, the analysis can be triggered using the following commands:
\begin{lstlisting}[style=bash]
<@\textcolor{magenta}{infer}@> capture -- gcc -c source_file.c
# Without the '-only' suffix, it is triggered with other Infer's analyses.
<@\textcolor{magenta}{infer}@> analyze --atomic-sets-only # --atomic-sets
<@\textcolor{magenta}{infer}@> analyze --atomicity-violations-only # --atomicity-violations
\end{lstlisting}
To analyse Java programs, use the following commands:
\begin{lstlisting}[style=bash]
<@\textcolor{magenta}{infer}@> capture -- javac source_file.java
<@\textcolor{magenta}{infer}@> analyze --atomic-sets-only # --atomic-sets
<@\textcolor{magenta}{infer}@> analyze --atomicity-violations-only # --atomicity-violations
\end{lstlisting}

\subsection*{Input Parameters}

The following parameters may be used when running the \texttt{infer analyze} command:
\begin{itemize}[label={}]
    \item \texttt{-{}-atomicity-ignored-function-calls path}
    \begin{itemize}
        \item Specify a~file with function names (one function name per line; considered as a~regular expression if the line starts with~\texttt{R} followed by whitespace, an exact match otherwise) whose \emph{calls should be ignored} during the analysis.
    \end{itemize}
    
    \item \texttt{-{}-atomicity-ignored-function-analyses path}
    \begin{itemize}
        \item Specify a~file with function names (one function name per line; considered as a~regular expression if the line starts with~\texttt{R} followed by whitespace, an exact match otherwise) whose \emph{analysis should be ignored} during the analysis.
    \end{itemize}
    
    \item \texttt{-{}-atomicity-allowed-function-calls path}
    \begin{itemize}
        \item Specify a~file with function names (one function name per line; considered as a~regular expression if the line starts with~\texttt{R} followed by whitespace, an exact match otherwise) whose \emph{calls should be allowed} during the analysis. Other functions will be ignored.
    \end{itemize}
    
    \item \texttt{-{}-atomicity-allowed-function-analyses path}
    \begin{itemize}
        \item Specify a~file with function names (one function name per line; considered as a~regular expression if the line starts with~\texttt{R} followed by whitespace, an exact match otherwise) whose \emph{analysis should be allowed} during the analysis. Other functions will be ignored.
    \end{itemize}
    
    \item \texttt{-{}-atomic-sets-locked-functions-limit int}
    \begin{itemize}
        \item Specify the \emph{maximum number of function calls} that could appear in a~\emph{critical section} in the first phase of the analysis. Critical sections with more function calls will be ignored. This is the parameter~$ d $. The default value is~20.
    \end{itemize}
    
    \item \texttt{-{}-atomic-sets-functions-depth-limit int}
    \begin{itemize}
        \item Specify the \emph{maximum depth in the hierarchy of function calls} to which function calls will be considered during the first phase of the analysis. This is the parameter~$ r $. The default value is~10.
    \end{itemize}
    
    \item \texttt{-{}-atomicity-lock-level-limit int}
    \begin{itemize}
        \item Specify the \emph{maximum expected level of ownership over the same lock object}. An over-approximation of the number of times the lock has been acquired. This is the parameter~$ t $. The default value is~5.
    \end{itemize}
\end{itemize}
For more information about the input parameters and other helpful parameters, see \texttt{man infer} or \texttt{man infer-analyze}.



%===============================================================================



\chapter{Atomer's Implementation Algorithms}
\label{app:alg}

This appendix provides an overview of some of the \emph{implementation algorithms} of the new version of Atomer. These algorithms are either too long or not that important to be placed in the main content of the thesis.


\section*{Phase~1\,---\,Detection of Atomic Sets}

This section deals with algorithms related to the \emph{first phase} of the analysis.

\vfill

\begin{algorithm}[H]
    \KwData{access paths $ L \in 2^\Pi $ of locks being unlocked; abstract state $ s \in \mathtt{Domain.t} $}
%
    \Def{\texttt{\upshape{apply\_unlocks($ L $, $ s $)}}}{%
        \For{$ p \in s $}{%
            \For{$ (A, B, \mathcal{L}) \in p.callsPairs $}{%
                \If(\tcc*[f]{unlock an existing lock}){$ \mathcal{L}.\pi \in L $}{%
                    $ \mathcal{L} \leftarrow \text{\texttt{\upshape{Lock.unlock}}}(\mathcal{L}) $\;
                    \If{$ \neg \text{\texttt{\upshape{Lock.is\_locked}}}(\mathcal{L}) $}{%
                        \tcp{the lock is \uv{completely} unlocked}
                        $ p.finalCallsPairs \leftarrow p.finalCallsPairs \cup \{(A, B)\} $\;
                        $ p.callsPairs \leftarrow p.callsPairs \setminus \{(A, B, \mathcal{L})\} $\;
                        $ p.calls \leftarrow \emptyset $\;
                    }
                }
            }
        }
        \Return{$ s $}\;
    }
%
    \caption{Updating an abstract state after an \emph{unlock}}
    \label{alg:phase1AppUnlock}
\end{algorithm}

\vfill

\clearpage

\vfill

\begin{algorithm}[H]
    \KwData{access paths $ L \in 2^\Pi $ of locks being locked; abstract state $ s \in \mathtt{Domain.t} $}
%
    \Def{\texttt{\upshape{apply\_locks($ L $, $ s $)}}}{%
        \For{$ p \in s $}{%
            $ lockCreated \leftarrow false $\;
            \For{$ \pi \in L $}{%
                $ locked \leftarrow false $\;
                \For{$ (A, B, \mathcal{L}) \in p.callsPairs $}{%
                    \If(\tcc*[f]{lock an existing lock}){$ \pi = \mathcal{L}.\pi $}{%
                        $ locked \leftarrow true $\;
                        $ \mathcal{L} \leftarrow \text{\texttt{\upshape{Lock.lock}}}(\mathcal{L}) $\;
                    }
                }
                \If(\tcc*[f]{create a~new lock}){$ \neg locked $}{%
                    $ lockCreated \leftarrow true $\;
                    $ p.callsPairs \leftarrow p.callsPairs \cup \{(p.calls, \emptyset, \text{\texttt{\upshape{Lock.create}}}(\pi))\} $\;
                }
            }
            \lIf{$ lockCreated $}{$ p.calls \leftarrow \emptyset $}
        }
        \Return{$ s $}\;
    }
%
    \caption{Updating an abstract state after a~\emph{lock}}
    \label{alg:phase1AppLock}
\end{algorithm}

\vfill

\begin{algorithm}[H]
    \KwData{summary $ \chi \in \mathtt{Domain.Summary.t} $ of a~called function; abstract state $ s \in \mathtt{Domain.t} $; maximum length $ d \in \mathbb{N} $ of a~critical section; the number $ r \in \mathbb{N} $ of levels for the consideration of nested calls}
%
    \Def{\texttt{\upshape{apply\_summary($ \chi $, $ s $)}}}{%
        \For{$ p \in s $}{%
            $ nestedCalls \leftarrow \emptyset $\;
            \For(\tcc*[f]{get nested calls from lower-levels}){$ (i, C) \in \chi.allCalls $}{%
                \If(\tcc*[f]{move the calls one level lower}){$ i + 1 < r $}{%
                    \lIf{$ \exists\,p.allCalls[i + 1] $}{$ p.allCalls[i + 1] \leftarrow p.allCalls[i + 1] \cup C $}
                    \lElse{$ p.allCalls[i + 1] \leftarrow C $}
                }
                \lIf{$ i < r $}{$ nestedCalls \leftarrow nestedCalls \cup C $}
            }
            $ p.calls \leftarrow p.calls \cup nestedCalls $\;
            \For{$ (A, B, \mathcal{L}) \in p.callsPairs $}{%
                $ B \leftarrow B \cup nestedCalls $\;
                \tcp{rid of \uv{large} critical sections}
                \lIf{$ |B| > d $}{$ p.callsPairs \leftarrow p.callsPairs \setminus \{(A, B, \mathcal{L})\} $}
            }
        }
        \Return{$ s $}\;
    }
%
    \caption{Updating an abstract state with a~\emph{summary of a~called function}}
    \label{alg:phase1ApplSumm}
\end{algorithm}

\vfill

\clearpage


\section*{Phase~2\,---\,Detection of Atomicity Violations}

This section deals with algorithms related to the \emph{second phase} of the analysis.

\vfill

\begin{algorithm}[H]
    \KwData{access paths $ L \in 2^\Pi $ of locks being unlocked; abstract state $ s \in \mathtt{Domain.t} $}
%
    \Def{\texttt{\upshape{apply\_unlocks($ L $, $ s $)}}}{%
        \For{$ p \in s $}{%
            \For{$ (\mathtt{x}, \mathtt{y}, \mathcal{L}) \in p.lockedLastPairs $}{%
                \If(\tcc*[f]{unlock an existing lock}){$ \mathcal{L}.\pi \in L $}{%
                    $ \mathcal{L} \leftarrow \text{\texttt{\upshape{Lock.unlock}}}(\mathcal{L}) $\;
                    \If{$ \neg \text{\texttt{\upshape{Lock.is\_locked}}}(\mathcal{L}) $}{%
                        \tcp{the lock is \uv{completely} unlocked}
                        $ p.lockedLastPairs \leftarrow p.lockedLastPairs \setminus \{(\mathtt{x}, \mathtt{y}, \mathcal{L})\} $\;
                    }
                }
            }
        }
        \Return{$ s $}\;
    }
%
    \caption{Updating an abstract state after an \emph{unlock}}
    \label{alg:phase2AppUnlock}
\end{algorithm}

\vfill

\begin{algorithm}[H]
    \KwData{access paths $ L \in 2^\Pi $ of locks being locked; abstract state $ s \in \mathtt{Domain.t} $}
%
    \Def{\texttt{\upshape{apply\_locks($ L $, $ s $)}}}{%
        \For{$ p \in s $}{%
            \For{$ \pi \in L $}{%
                $ locked \leftarrow false $\;
                \For{$ (\mathtt{x}, \mathtt{y}, \mathcal{L}) \in p.lockedLastPairs $}{%
                    \If(\tcc*[f]{lock an existing lock}){$ \pi = \mathcal{L}.\pi $}{%
                        $ locked \leftarrow true $\;
                        $ \mathcal{L} \leftarrow \text{\texttt{\upshape{Lock.lock}}}(\mathcal{L}) $\;
                    }
                }
                \If(\tcc*[f]{create a~new lock}){$ \neg locked $}{%
                    $ p.lockedLastPairs \leftarrow p.lockedLastPairs \cup \{(\varepsilon, \varepsilon, \text{\texttt{\upshape{Lock.create}}}(\pi))\} $\;
                }
            }
        }
        \Return{$ s $}\;
    }
%
    \caption{Updating an abstract state after a~\emph{lock}}
    \label{alg:phase2AppLock}
\end{algorithm}

\vfill

\clearpage

\leavevmode\vfill

\begin{algorithm}[H]
    \KwData{summary $ \chi \in \mathtt{Domain.Summary.t} $ of a~called function; abstract state $ s \in \mathtt{Domain.t} $}
%
    \Def{\texttt{\upshape{apply\_summary($ \chi $, $ s $)}}}{%
        \For{$ p \in s $}{%
            \eIf(\tcc*[f]{outside an atomic section}){$ p.lockedLastPairs = \emptyset $}{%
                \tcp{take summary violations unchanged}
                $ p.violations \leftarrow p.violations \cup \chi.violations $\;
            }(\tcc*[f]{inside an atomic section}){%
                \tcp{take summary violations as warnings}
                \For{$ (\mathtt{x}, \mathtt{y}, c, S) \in \chi.violations $}{%
                    $ p.violations \leftarrow p.violations \cup \{(\mathtt{x}, \mathtt{y}, c, \mathtt{Warning})\} $\;
                }
            }
        }
        \Return{$ s $}\;
    }
%
    \caption{Updating an abstract state with a~\emph{summary of a~called function}}
    \label{alg:phase2ApplSumm}
\end{algorithm}

\vfill



%===============================================================================
